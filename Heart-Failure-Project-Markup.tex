% Options for packages loaded elsewhere
\PassOptionsToPackage{unicode}{hyperref}
\PassOptionsToPackage{hyphens}{url}
%
\documentclass[
]{article}
\usepackage{amsmath,amssymb}
\usepackage{lmodern}
\usepackage{ifxetex,ifluatex}
\ifnum 0\ifxetex 1\fi\ifluatex 1\fi=0 % if pdftex
  \usepackage[T1]{fontenc}
  \usepackage[utf8]{inputenc}
  \usepackage{textcomp} % provide euro and other symbols
\else % if luatex or xetex
  \usepackage{unicode-math}
  \defaultfontfeatures{Scale=MatchLowercase}
  \defaultfontfeatures[\rmfamily]{Ligatures=TeX,Scale=1}
\fi
% Use upquote if available, for straight quotes in verbatim environments
\IfFileExists{upquote.sty}{\usepackage{upquote}}{}
\IfFileExists{microtype.sty}{% use microtype if available
  \usepackage[]{microtype}
  \UseMicrotypeSet[protrusion]{basicmath} % disable protrusion for tt fonts
}{}
\makeatletter
\@ifundefined{KOMAClassName}{% if non-KOMA class
  \IfFileExists{parskip.sty}{%
    \usepackage{parskip}
  }{% else
    \setlength{\parindent}{0pt}
    \setlength{\parskip}{6pt plus 2pt minus 1pt}}
}{% if KOMA class
  \KOMAoptions{parskip=half}}
\makeatother
\usepackage{xcolor}
\IfFileExists{xurl.sty}{\usepackage{xurl}}{} % add URL line breaks if available
\IfFileExists{bookmark.sty}{\usepackage{bookmark}}{\usepackage{hyperref}}
\hypersetup{
  pdftitle={Machine Learning Project - Kaggle Heart Failure},
  hidelinks,
  pdfcreator={LaTeX via pandoc}}
\urlstyle{same} % disable monospaced font for URLs
\usepackage[margin=1in]{geometry}
\usepackage{color}
\usepackage{fancyvrb}
\newcommand{\VerbBar}{|}
\newcommand{\VERB}{\Verb[commandchars=\\\{\}]}
\DefineVerbatimEnvironment{Highlighting}{Verbatim}{commandchars=\\\{\}}
% Add ',fontsize=\small' for more characters per line
\usepackage{framed}
\definecolor{shadecolor}{RGB}{248,248,248}
\newenvironment{Shaded}{\begin{snugshade}}{\end{snugshade}}
\newcommand{\AlertTok}[1]{\textcolor[rgb]{0.94,0.16,0.16}{#1}}
\newcommand{\AnnotationTok}[1]{\textcolor[rgb]{0.56,0.35,0.01}{\textbf{\textit{#1}}}}
\newcommand{\AttributeTok}[1]{\textcolor[rgb]{0.77,0.63,0.00}{#1}}
\newcommand{\BaseNTok}[1]{\textcolor[rgb]{0.00,0.00,0.81}{#1}}
\newcommand{\BuiltInTok}[1]{#1}
\newcommand{\CharTok}[1]{\textcolor[rgb]{0.31,0.60,0.02}{#1}}
\newcommand{\CommentTok}[1]{\textcolor[rgb]{0.56,0.35,0.01}{\textit{#1}}}
\newcommand{\CommentVarTok}[1]{\textcolor[rgb]{0.56,0.35,0.01}{\textbf{\textit{#1}}}}
\newcommand{\ConstantTok}[1]{\textcolor[rgb]{0.00,0.00,0.00}{#1}}
\newcommand{\ControlFlowTok}[1]{\textcolor[rgb]{0.13,0.29,0.53}{\textbf{#1}}}
\newcommand{\DataTypeTok}[1]{\textcolor[rgb]{0.13,0.29,0.53}{#1}}
\newcommand{\DecValTok}[1]{\textcolor[rgb]{0.00,0.00,0.81}{#1}}
\newcommand{\DocumentationTok}[1]{\textcolor[rgb]{0.56,0.35,0.01}{\textbf{\textit{#1}}}}
\newcommand{\ErrorTok}[1]{\textcolor[rgb]{0.64,0.00,0.00}{\textbf{#1}}}
\newcommand{\ExtensionTok}[1]{#1}
\newcommand{\FloatTok}[1]{\textcolor[rgb]{0.00,0.00,0.81}{#1}}
\newcommand{\FunctionTok}[1]{\textcolor[rgb]{0.00,0.00,0.00}{#1}}
\newcommand{\ImportTok}[1]{#1}
\newcommand{\InformationTok}[1]{\textcolor[rgb]{0.56,0.35,0.01}{\textbf{\textit{#1}}}}
\newcommand{\KeywordTok}[1]{\textcolor[rgb]{0.13,0.29,0.53}{\textbf{#1}}}
\newcommand{\NormalTok}[1]{#1}
\newcommand{\OperatorTok}[1]{\textcolor[rgb]{0.81,0.36,0.00}{\textbf{#1}}}
\newcommand{\OtherTok}[1]{\textcolor[rgb]{0.56,0.35,0.01}{#1}}
\newcommand{\PreprocessorTok}[1]{\textcolor[rgb]{0.56,0.35,0.01}{\textit{#1}}}
\newcommand{\RegionMarkerTok}[1]{#1}
\newcommand{\SpecialCharTok}[1]{\textcolor[rgb]{0.00,0.00,0.00}{#1}}
\newcommand{\SpecialStringTok}[1]{\textcolor[rgb]{0.31,0.60,0.02}{#1}}
\newcommand{\StringTok}[1]{\textcolor[rgb]{0.31,0.60,0.02}{#1}}
\newcommand{\VariableTok}[1]{\textcolor[rgb]{0.00,0.00,0.00}{#1}}
\newcommand{\VerbatimStringTok}[1]{\textcolor[rgb]{0.31,0.60,0.02}{#1}}
\newcommand{\WarningTok}[1]{\textcolor[rgb]{0.56,0.35,0.01}{\textbf{\textit{#1}}}}
\usepackage{graphicx}
\makeatletter
\def\maxwidth{\ifdim\Gin@nat@width>\linewidth\linewidth\else\Gin@nat@width\fi}
\def\maxheight{\ifdim\Gin@nat@height>\textheight\textheight\else\Gin@nat@height\fi}
\makeatother
% Scale images if necessary, so that they will not overflow the page
% margins by default, and it is still possible to overwrite the defaults
% using explicit options in \includegraphics[width, height, ...]{}
\setkeys{Gin}{width=\maxwidth,height=\maxheight,keepaspectratio}
% Set default figure placement to htbp
\makeatletter
\def\fps@figure{htbp}
\makeatother
\setlength{\emergencystretch}{3em} % prevent overfull lines
\providecommand{\tightlist}{%
  \setlength{\itemsep}{0pt}\setlength{\parskip}{0pt}}
\setcounter{secnumdepth}{-\maxdimen} % remove section numbering
\ifluatex
  \usepackage{selnolig}  % disable illegal ligatures
\fi

\title{Machine Learning Project - Kaggle Heart Failure}
\author{}
\date{\vspace{-2.5em}}

\begin{document}
\maketitle

\hypertarget{heart-failure-prediction}{%
\section{HEART FAILURE PREDICTION}\label{heart-failure-prediction}}

This project involves using heart failure data from different patients
to create a machine learning model which can be used for predicting
whether a patient is likely to have a fatal heart failure event. There
are several steps taken in this project. Step 1 involves loading,
visualising and transforming the data, step 2 involves splitting the
data into a training and test set which will be used during the modeling
steps. Step 3 involves trying out different models and comparing the
accuracy, ROC plot and AUC. Step 4 is short listing the best models
which will then be further tuned to improve the performance
(hyperparameters), step 5 will involve feature engineering to attempt to
further improve performance of the models. A final best model will be
chosen.

\hypertarget{step-1-adding-required-libraries-loading-dataset-transforming-and-visualling-data--}{%
\subsection{STEP 1: Adding required libraries, loading dataset,
transforming and visualling data
-}\label{step-1-adding-required-libraries-loading-dataset-transforming-and-visualling-data--}}

\begin{Shaded}
\begin{Highlighting}[]
\FunctionTok{library}\NormalTok{(caret)}
\end{Highlighting}
\end{Shaded}

\begin{verbatim}
## Loading required package: lattice
\end{verbatim}

\begin{verbatim}
## Loading required package: ggplot2
\end{verbatim}

\begin{Shaded}
\begin{Highlighting}[]
\FunctionTok{library}\NormalTok{(tidyverse)}
\end{Highlighting}
\end{Shaded}

\begin{verbatim}
## -- Attaching packages --------------------------------------- tidyverse 1.3.0 --
\end{verbatim}

\begin{verbatim}
## v tibble  3.1.0     v dplyr   1.0.4
## v tidyr   1.1.2     v stringr 1.4.0
## v readr   1.4.0     v forcats 0.5.1
## v purrr   0.3.4
\end{verbatim}

\begin{verbatim}
## -- Conflicts ------------------------------------------ tidyverse_conflicts() --
## x dplyr::filter() masks stats::filter()
## x dplyr::lag()    masks stats::lag()
## x purrr::lift()   masks caret::lift()
\end{verbatim}

\begin{Shaded}
\begin{Highlighting}[]
\FunctionTok{library}\NormalTok{(corrplot)}
\end{Highlighting}
\end{Shaded}

\begin{verbatim}
## corrplot 0.84 loaded
\end{verbatim}

\begin{Shaded}
\begin{Highlighting}[]
\FunctionTok{library}\NormalTok{(ggplot2)}
\FunctionTok{library}\NormalTok{(pROC)}
\end{Highlighting}
\end{Shaded}

\begin{verbatim}
## Type 'citation("pROC")' for a citation.
\end{verbatim}

\begin{verbatim}
## 
## Attaching package: 'pROC'
\end{verbatim}

\begin{verbatim}
## The following objects are masked from 'package:stats':
## 
##     cov, smooth, var
\end{verbatim}

\begin{Shaded}
\begin{Highlighting}[]
\FunctionTok{library}\NormalTok{(ROCR)}
\FunctionTok{library}\NormalTok{(xgboost)}
\end{Highlighting}
\end{Shaded}

\begin{verbatim}
## 
## Attaching package: 'xgboost'
\end{verbatim}

\begin{verbatim}
## The following object is masked from 'package:dplyr':
## 
##     slice
\end{verbatim}

\begin{Shaded}
\begin{Highlighting}[]
\FunctionTok{library}\NormalTok{(rattle)}
\end{Highlighting}
\end{Shaded}

\begin{verbatim}
## Loading required package: bitops
\end{verbatim}

\begin{verbatim}
## Rattle: A free graphical interface for data science with R.
## Version 5.4.0 Copyright (c) 2006-2020 Togaware Pty Ltd.
## Type 'rattle()' to shake, rattle, and roll your data.
\end{verbatim}

\begin{verbatim}
## 
## Attaching package: 'rattle'
\end{verbatim}

\begin{verbatim}
## The following object is masked from 'package:xgboost':
## 
##     xgboost
\end{verbatim}

\begin{Shaded}
\begin{Highlighting}[]
\FunctionTok{library}\NormalTok{(dplyr)}
\end{Highlighting}
\end{Shaded}

The required libraries are imported.

\begin{Shaded}
\begin{Highlighting}[]
\NormalTok{file\_path }\OtherTok{\textless{}{-}} \StringTok{"\textasciitilde{}/Desktop/Graduate Life/Machine Learning/Kaggle Datasets"}
\NormalTok{raw\_dataset }\OtherTok{\textless{}{-}} \FunctionTok{read.csv}\NormalTok{(}\FunctionTok{paste}\NormalTok{(file\_path,}\StringTok{"heart\_failure\_clinical\_records\_dataset.csv"}\NormalTok{,}\AttributeTok{sep=}\StringTok{"/"}\NormalTok{), }\AttributeTok{header=}\ConstantTok{TRUE}\NormalTok{) }\CommentTok{\#Load CSV}
\FunctionTok{glimpse}\NormalTok{(raw\_dataset)}
\end{Highlighting}
\end{Shaded}

\begin{verbatim}
## Rows: 299
## Columns: 13
## $ age                      <dbl> 75, 55, 65, 50, 65, 90, 75, 60, 65, 80, 75, 6~
## $ anaemia                  <int> 0, 0, 0, 1, 1, 1, 1, 1, 0, 1, 1, 0, 1, 1, 1, ~
## $ creatinine_phosphokinase <int> 582, 7861, 146, 111, 160, 47, 246, 315, 157, ~
## $ diabetes                 <int> 0, 0, 0, 0, 1, 0, 0, 1, 0, 0, 0, 0, 0, 0, 0, ~
## $ ejection_fraction        <int> 20, 38, 20, 20, 20, 40, 15, 60, 65, 35, 38, 2~
## $ high_blood_pressure      <int> 1, 0, 0, 0, 0, 1, 0, 0, 0, 1, 1, 1, 0, 1, 1, ~
## $ platelets                <dbl> 265000, 263358, 162000, 210000, 327000, 20400~
## $ serum_creatinine         <dbl> 1.90, 1.10, 1.30, 1.90, 2.70, 2.10, 1.20, 1.1~
## $ serum_sodium             <int> 130, 136, 129, 137, 116, 132, 137, 131, 138, ~
## $ sex                      <int> 1, 1, 1, 1, 0, 1, 1, 1, 0, 1, 1, 1, 1, 1, 0, ~
## $ smoking                  <int> 0, 0, 1, 0, 0, 1, 0, 1, 0, 1, 1, 1, 0, 0, 0, ~
## $ time                     <int> 4, 6, 7, 7, 8, 8, 10, 10, 10, 10, 10, 10, 11,~
## $ DEATH_EVENT              <int> 1, 1, 1, 1, 1, 1, 1, 1, 1, 1, 1, 1, 1, 1, 0, ~
\end{verbatim}

The kaggle dataset for the heart failure clinical records is the loaded
and a glimpse of the dataset is presented. Here we can see that there
are binary features which are currently of the wrong datatype (should be
factors). This is fixed by mutating the columns. There are also 299
patient data rows made up of 13 columns (12 feature variables and 1
response variable).

\begin{Shaded}
\begin{Highlighting}[]
\NormalTok{dataset }\OtherTok{\textless{}{-}}\NormalTok{ raw\_dataset }\SpecialCharTok{\%\textgreater{}\%}
  \FunctionTok{mutate}\NormalTok{(}\AttributeTok{anaemia =} \FunctionTok{if\_else}\NormalTok{(anaemia}\SpecialCharTok{==}\DecValTok{1}\NormalTok{,}\StringTok{"YES"}\NormalTok{,}\StringTok{"NO"}\NormalTok{),}
         \AttributeTok{diabetes =} \FunctionTok{if\_else}\NormalTok{(diabetes}\SpecialCharTok{==}\DecValTok{1}\NormalTok{,}\StringTok{"YES"}\NormalTok{,}\StringTok{"NO"}\NormalTok{),}
         \AttributeTok{high\_blood\_pressure =} \FunctionTok{if\_else}\NormalTok{(high\_blood\_pressure}\SpecialCharTok{==}\DecValTok{1}\NormalTok{,}\StringTok{"YES"}\NormalTok{,}\StringTok{"NO"}\NormalTok{),}
         \AttributeTok{sex =} \FunctionTok{if\_else}\NormalTok{(sex}\SpecialCharTok{==}\DecValTok{1}\NormalTok{,}\StringTok{"MALE"}\NormalTok{,}\StringTok{"FEMALE"}\NormalTok{),}
         \AttributeTok{smoking =} \FunctionTok{if\_else}\NormalTok{(smoking}\SpecialCharTok{==}\DecValTok{1}\NormalTok{,}\StringTok{"YES"}\NormalTok{,}\StringTok{"NO"}\NormalTok{),}
         \AttributeTok{DEATH\_EVENT =} \FunctionTok{if\_else}\NormalTok{(DEATH\_EVENT}\SpecialCharTok{==}\DecValTok{1}\NormalTok{,}\StringTok{"YES"}\NormalTok{,}\StringTok{"NO"}\NormalTok{),}
         \AttributeTok{age =} \FunctionTok{as.integer}\NormalTok{(age),}
         \AttributeTok{platelets =} \FunctionTok{as.integer}\NormalTok{(platelets)}
\NormalTok{  ) }\SpecialCharTok{\%\textgreater{}\%}
  \FunctionTok{mutate\_if}\NormalTok{(is.character,as.factor) }\SpecialCharTok{\%\textgreater{}\%}
\NormalTok{  dplyr}\SpecialCharTok{::}\FunctionTok{select}\NormalTok{(DEATH\_EVENT,anaemia,diabetes,high\_blood\_pressure,sex,smoking,}\FunctionTok{everything}\NormalTok{())}
\end{Highlighting}
\end{Shaded}

The features `anaemia', `diabetes', high\_blood\_pressure', `sex',
`smoking', `DEATH\_EVENT' are converted to factors where 1 = `YES' and 0
= `NO'. `Sex' is also converted to a factor with 1 = `MALE' and 0 =
`FEMALE'. Finally `age' and `platelets' are converted to intergers as
they are currently doubles.

We then glimpse the transformed dataset to ensure this has worked and
then have a look at the first 6 rows of the dataset.

\begin{Shaded}
\begin{Highlighting}[]
\FunctionTok{glimpse}\NormalTok{(dataset)}
\end{Highlighting}
\end{Shaded}

\begin{verbatim}
## Rows: 299
## Columns: 13
## $ DEATH_EVENT              <fct> YES, YES, YES, YES, YES, YES, YES, YES, YES, ~
## $ anaemia                  <fct> NO, NO, NO, YES, YES, YES, YES, YES, NO, YES,~
## $ diabetes                 <fct> NO, NO, NO, NO, YES, NO, NO, YES, NO, NO, NO,~
## $ high_blood_pressure      <fct> YES, NO, NO, NO, NO, YES, NO, NO, NO, YES, YE~
## $ sex                      <fct> MALE, MALE, MALE, MALE, FEMALE, MALE, MALE, M~
## $ smoking                  <fct> NO, NO, YES, NO, NO, YES, NO, YES, NO, YES, Y~
## $ age                      <int> 75, 55, 65, 50, 65, 90, 75, 60, 65, 80, 75, 6~
## $ creatinine_phosphokinase <int> 582, 7861, 146, 111, 160, 47, 246, 315, 157, ~
## $ ejection_fraction        <int> 20, 38, 20, 20, 20, 40, 15, 60, 65, 35, 38, 2~
## $ platelets                <int> 265000, 263358, 162000, 210000, 327000, 20400~
## $ serum_creatinine         <dbl> 1.90, 1.10, 1.30, 1.90, 2.70, 2.10, 1.20, 1.1~
## $ serum_sodium             <int> 130, 136, 129, 137, 116, 132, 137, 131, 138, ~
## $ time                     <int> 4, 6, 7, 7, 8, 8, 10, 10, 10, 10, 10, 10, 11,~
\end{verbatim}

\begin{Shaded}
\begin{Highlighting}[]
\FunctionTok{head}\NormalTok{(dataset)}
\end{Highlighting}
\end{Shaded}

\begin{verbatim}
##   DEATH_EVENT anaemia diabetes high_blood_pressure    sex smoking age
## 1         YES      NO       NO                 YES   MALE      NO  75
## 2         YES      NO       NO                  NO   MALE      NO  55
## 3         YES      NO       NO                  NO   MALE     YES  65
## 4         YES     YES       NO                  NO   MALE      NO  50
## 5         YES     YES      YES                  NO FEMALE      NO  65
## 6         YES     YES       NO                 YES   MALE     YES  90
##   creatinine_phosphokinase ejection_fraction platelets serum_creatinine
## 1                      582                20    265000              1.9
## 2                     7861                38    263358              1.1
## 3                      146                20    162000              1.3
## 4                      111                20    210000              1.9
## 5                      160                20    327000              2.7
## 6                       47                40    204000              2.1
##   serum_sodium time
## 1          130    4
## 2          136    6
## 3          129    7
## 4          137    7
## 5          116    8
## 6          132    8
\end{verbatim}

We can take a look at the percentage breakdown of those who have had a
fatal heart failure vs those patients who have not. Here we see approx.
32\% of patients suffered a fatal event, whilst 68\% did not.

\begin{Shaded}
\begin{Highlighting}[]
\NormalTok{percentage }\OtherTok{\textless{}{-}} \FunctionTok{prop.table}\NormalTok{(}\FunctionTok{table}\NormalTok{(dataset}\SpecialCharTok{$}\NormalTok{DEATH\_EVENT)) }\SpecialCharTok{*} \DecValTok{100}
\FunctionTok{cbind}\NormalTok{(}\AttributeTok{freq=}\FunctionTok{table}\NormalTok{(dataset}\SpecialCharTok{$}\NormalTok{DEATH\_EVENT), }\AttributeTok{percentage=}\NormalTok{percentage)}
\end{Highlighting}
\end{Shaded}

\begin{verbatim}
##     freq percentage
## NO   203   67.89298
## YES   96   32.10702
\end{verbatim}

A correlation plot of the non-categorical variables shows there is not
much correlation between the different features.

\begin{Shaded}
\begin{Highlighting}[]
\NormalTok{corr\_data }\OtherTok{\textless{}{-}} \FunctionTok{cor}\NormalTok{(dataset[}\DecValTok{7}\SpecialCharTok{:}\DecValTok{13}\NormalTok{])}
\FunctionTok{corrplot}\NormalTok{(corr\_data, }\AttributeTok{order=}\StringTok{"hclust"}\NormalTok{, }\AttributeTok{col=}\FunctionTok{c}\NormalTok{(}\StringTok{"black"}\NormalTok{, }\StringTok{"white"}\NormalTok{), }\AttributeTok{bg=}\StringTok{"lightblue"}\NormalTok{, }\AttributeTok{type=}\StringTok{"upper"}\NormalTok{)}
\end{Highlighting}
\end{Shaded}

\includegraphics{Heart-Failure-Project-Markup_files/figure-latex/unnamed-chunk-6-1.pdf}

This can further seen in the plots below which show that there is little
correlation between the different continuous numerical features.

\begin{Shaded}
\begin{Highlighting}[]
\FunctionTok{plot}\NormalTok{(dataset[}\DecValTok{7}\SpecialCharTok{:}\DecValTok{13}\NormalTok{])}
\end{Highlighting}
\end{Shaded}

\includegraphics{Heart-Failure-Project-Markup_files/figure-latex/unnamed-chunk-7-1.pdf}

\ldots{}

\hypertarget{step-2-splitting-dataset-into-training-and-test-sets.}{%
\subsection{STEP 2: splitting dataset into training and test
sets.}\label{step-2-splitting-dataset-into-training-and-test-sets.}}

The dataset is then split into a training sample and a test sample. The
training:test split will be 80:20. The training data will be exclusively
used to train the different models (using K Fold Cross Validation) and
the test data is used exclusively for measuring the performance of the
models.

\begin{Shaded}
\begin{Highlighting}[]
\FunctionTok{set.seed}\NormalTok{(}\DecValTok{1995}\NormalTok{)}
\NormalTok{training\_samples }\OtherTok{\textless{}{-}} \FunctionTok{createDataPartition}\NormalTok{(dataset}\SpecialCharTok{$}\NormalTok{DEATH\_EVENT, }\AttributeTok{p=}\FloatTok{0.80}\NormalTok{, }\AttributeTok{list=}\ConstantTok{FALSE}\NormalTok{)}
\NormalTok{train\_data }\OtherTok{\textless{}{-}}\NormalTok{ dataset[training\_samples,]}
\NormalTok{test\_data }\OtherTok{\textless{}{-}}\NormalTok{ dataset[}\SpecialCharTok{{-}}\NormalTok{training\_samples,]}
\end{Highlighting}
\end{Shaded}

\hypertarget{step-3-training-the-different-models}{%
\subsection{STEP 3: Training the different
models}\label{step-3-training-the-different-models}}

Here we will go through and train each of the models we will compare for
the classification of heart failure. The models we will be using are -
Logistic Regression,

We have selected `age', `ejection\_fraction', `serum\_creatinine' and
`time' as these were shown to lead to the best accuracy in the prior run
and also have the highest importance in the importance plots -

\begin{Shaded}
\begin{Highlighting}[]
\FunctionTok{set.seed}\NormalTok{(}\DecValTok{692}\NormalTok{)}
\NormalTok{train\_control }\OtherTok{\textless{}{-}} \FunctionTok{trainControl}\NormalTok{(}\AttributeTok{method=}\StringTok{"repeatedcv"}\NormalTok{, }\AttributeTok{number=}\DecValTok{10}\NormalTok{)}
\NormalTok{log\_model }\OtherTok{\textless{}{-}} \FunctionTok{train}\NormalTok{(DEATH\_EVENT}\SpecialCharTok{\textasciitilde{}}\NormalTok{., }\AttributeTok{data=}\NormalTok{train\_data, }\AttributeTok{trControl=}\NormalTok{train\_control, }\AttributeTok{method=}\StringTok{"glm"}\NormalTok{, }\AttributeTok{family=}\StringTok{"binomial"}\NormalTok{)}
\NormalTok{importance }\OtherTok{\textless{}{-}} \FunctionTok{varImp}\NormalTok{(log\_model, }\AttributeTok{scale=}\ConstantTok{FALSE}\NormalTok{)}
\FunctionTok{plot}\NormalTok{(importance)}
\end{Highlighting}
\end{Shaded}

\includegraphics{Heart-Failure-Project-Markup_files/figure-latex/unnamed-chunk-9-1.pdf}

Logistic Regression:

\begin{Shaded}
\begin{Highlighting}[]
\FunctionTok{set.seed}\NormalTok{(}\DecValTok{692}\NormalTok{)}
\NormalTok{train\_control }\OtherTok{\textless{}{-}} \FunctionTok{trainControl}\NormalTok{(}\AttributeTok{method=}\StringTok{"repeatedcv"}\NormalTok{, }\AttributeTok{number=}\DecValTok{10}\NormalTok{)}
\NormalTok{log\_model }\OtherTok{\textless{}{-}} \FunctionTok{train}\NormalTok{(DEATH\_EVENT}\SpecialCharTok{\textasciitilde{}}\NormalTok{ age }\SpecialCharTok{+}\NormalTok{ ejection\_fraction }\SpecialCharTok{+}\NormalTok{ serum\_creatinine }\SpecialCharTok{+}\NormalTok{ time, }
                   \AttributeTok{data=}\NormalTok{train\_data, }\AttributeTok{trControl=}\NormalTok{train\_control, }\AttributeTok{method=}\StringTok{"glm"}\NormalTok{, }\AttributeTok{family=}\StringTok{"binomial"}\NormalTok{)}
\NormalTok{log\_pred }\OtherTok{\textless{}{-}} \FunctionTok{predict}\NormalTok{(log\_model,}\AttributeTok{newdata =}\NormalTok{ test\_data)}
\end{Highlighting}
\end{Shaded}

K Nearest Neighbors:

\begin{Shaded}
\begin{Highlighting}[]
\FunctionTok{set.seed}\NormalTok{(}\DecValTok{200}\NormalTok{)}
\NormalTok{train\_control }\OtherTok{\textless{}{-}} \FunctionTok{trainControl}\NormalTok{(}\AttributeTok{method=}\StringTok{"repeatedcv"}\NormalTok{, }\AttributeTok{number=}\DecValTok{10}\NormalTok{)}
\NormalTok{knn\_model }\OtherTok{\textless{}{-}} \FunctionTok{train}\NormalTok{(DEATH\_EVENT }\SpecialCharTok{\textasciitilde{}}\NormalTok{ age }\SpecialCharTok{+}\NormalTok{ ejection\_fraction }\SpecialCharTok{+}\NormalTok{ serum\_creatinine }\SpecialCharTok{+}\NormalTok{ time, }\AttributeTok{data =}\NormalTok{ train\_data, }\AttributeTok{method =} \StringTok{"knn"}\NormalTok{, }\AttributeTok{trControl=}\NormalTok{train\_control, }\AttributeTok{preProcess =} \FunctionTok{c}\NormalTok{(}\StringTok{"center"}\NormalTok{, }\StringTok{"scale"}\NormalTok{), }\AttributeTok{tuneLength =} \DecValTok{10}\NormalTok{)}
\NormalTok{knn\_pred }\OtherTok{\textless{}{-}} \FunctionTok{predict}\NormalTok{(knn\_model, }\AttributeTok{newdata =}\NormalTok{ test\_data)}
\end{Highlighting}
\end{Shaded}

Support Vector Machine:

\begin{Shaded}
\begin{Highlighting}[]
\FunctionTok{set.seed}\NormalTok{(}\DecValTok{723}\NormalTok{)}
\NormalTok{train\_control }\OtherTok{\textless{}{-}} \FunctionTok{trainControl}\NormalTok{(}\AttributeTok{method=}\StringTok{"repeatedcv"}\NormalTok{, }\AttributeTok{number=}\DecValTok{10}\NormalTok{, }\AttributeTok{classProbs =}  \ConstantTok{TRUE}\NormalTok{)}
\NormalTok{svm\_model }\OtherTok{\textless{}{-}} \FunctionTok{train}\NormalTok{(DEATH\_EVENT }\SpecialCharTok{\textasciitilde{}}\NormalTok{age }\SpecialCharTok{+}\NormalTok{ ejection\_fraction }\SpecialCharTok{+}\NormalTok{ serum\_creatinine }\SpecialCharTok{+}\NormalTok{ time, }\AttributeTok{data =}\NormalTok{ train\_data, }\AttributeTok{method =} \StringTok{"svmLinear"}\NormalTok{, }\AttributeTok{trControl =}\NormalTok{ train\_control, }\AttributeTok{preProcess =} \FunctionTok{c}\NormalTok{(}\StringTok{"center"}\NormalTok{,}\StringTok{"scale"}\NormalTok{))}
\NormalTok{svm\_pred }\OtherTok{\textless{}{-}} \FunctionTok{predict}\NormalTok{(svm\_model, }\AttributeTok{newdata =}\NormalTok{ test\_data)}
\end{Highlighting}
\end{Shaded}

SVM using Non-Linear Kernel

\begin{Shaded}
\begin{Highlighting}[]
\FunctionTok{set.seed}\NormalTok{(}\DecValTok{139}\NormalTok{)}
\NormalTok{train\_control }\OtherTok{\textless{}{-}} \FunctionTok{trainControl}\NormalTok{(}\AttributeTok{method=}\StringTok{"repeatedcv"}\NormalTok{, }\AttributeTok{number=}\DecValTok{10}\NormalTok{, }\AttributeTok{classProbs =}  \ConstantTok{TRUE}\NormalTok{)}
\NormalTok{non\_svm\_model }\OtherTok{\textless{}{-}} \FunctionTok{train}\NormalTok{(DEATH\_EVENT}\SpecialCharTok{\textasciitilde{}}\NormalTok{ age }\SpecialCharTok{+}\NormalTok{ ejection\_fraction }\SpecialCharTok{+}\NormalTok{ serum\_creatinine }\SpecialCharTok{+}\NormalTok{ time, }\AttributeTok{data=}\NormalTok{train\_data, }\AttributeTok{trControl=}\NormalTok{train\_control, }\AttributeTok{method=}\StringTok{"svmRadial"}\NormalTok{, }\AttributeTok{preProcess=}\FunctionTok{c}\NormalTok{(}\StringTok{"center"}\NormalTok{, }\StringTok{"scale"}\NormalTok{), }\AttributeTok{tunelength=}\DecValTok{10}\NormalTok{)}
\NormalTok{non\_svm\_pred }\OtherTok{\textless{}{-}} \FunctionTok{predict}\NormalTok{(non\_svm\_model, }\AttributeTok{newdata =}\NormalTok{ test\_data)}
\end{Highlighting}
\end{Shaded}

Decision Tree:

\begin{Shaded}
\begin{Highlighting}[]
\FunctionTok{set.seed}\NormalTok{(}\DecValTok{823}\NormalTok{)}
\NormalTok{train\_control }\OtherTok{\textless{}{-}} \FunctionTok{trainControl}\NormalTok{(}\AttributeTok{method=}\StringTok{"repeatedcv"}\NormalTok{, }\AttributeTok{number=}\DecValTok{10}\NormalTok{)}
\NormalTok{tree\_model }\OtherTok{\textless{}{-}} \FunctionTok{train}\NormalTok{(DEATH\_EVENT}\SpecialCharTok{\textasciitilde{}}\NormalTok{age }\SpecialCharTok{+}\NormalTok{ ejection\_fraction }\SpecialCharTok{+}\NormalTok{ serum\_creatinine }\SpecialCharTok{+}\NormalTok{ time, }\AttributeTok{data=}\NormalTok{train\_data, }\AttributeTok{trControl=}\NormalTok{train\_control, }\AttributeTok{method=}\StringTok{"rpart"}\NormalTok{)}
\FunctionTok{fancyRpartPlot}\NormalTok{(tree\_model}\SpecialCharTok{$}\NormalTok{finalModel)}
\end{Highlighting}
\end{Shaded}

\includegraphics{Heart-Failure-Project-Markup_files/figure-latex/unnamed-chunk-14-1.pdf}

\begin{Shaded}
\begin{Highlighting}[]
\NormalTok{tree\_pred }\OtherTok{\textless{}{-}} \FunctionTok{predict}\NormalTok{(tree\_model, }\AttributeTok{newdata =}\NormalTok{ test\_data)}
\end{Highlighting}
\end{Shaded}

Random Forest Ensemble:

\begin{Shaded}
\begin{Highlighting}[]
\FunctionTok{set.seed}\NormalTok{(}\DecValTok{536}\NormalTok{)}
\NormalTok{train\_control }\OtherTok{\textless{}{-}} \FunctionTok{trainControl}\NormalTok{(}\AttributeTok{method=}\StringTok{"repeatedcv"}\NormalTok{, }\AttributeTok{number=}\DecValTok{10}\NormalTok{)}
\NormalTok{rf\_model }\OtherTok{\textless{}{-}} \FunctionTok{train}\NormalTok{(DEATH\_EVENT}\SpecialCharTok{\textasciitilde{}}\NormalTok{., }\AttributeTok{data=}\NormalTok{train\_data, }\AttributeTok{trControl=}\NormalTok{train\_control, }\AttributeTok{method=}\StringTok{"rf"}\NormalTok{)}
\NormalTok{rf\_pred }\OtherTok{\textless{}{-}} \FunctionTok{predict}\NormalTok{(rf\_model,}\AttributeTok{newdata =}\NormalTok{ test\_data)}
\end{Highlighting}
\end{Shaded}

Stochastic Gradient Boosting:

\begin{Shaded}
\begin{Highlighting}[]
\FunctionTok{set.seed}\NormalTok{(}\DecValTok{629}\NormalTok{)}
\NormalTok{train\_control }\OtherTok{\textless{}{-}} \FunctionTok{trainControl}\NormalTok{(}\AttributeTok{method=}\StringTok{"repeatedcv"}\NormalTok{, }\AttributeTok{number=}\DecValTok{10}\NormalTok{)}
\NormalTok{gbm\_model }\OtherTok{\textless{}{-}} \FunctionTok{train}\NormalTok{(DEATH\_EVENT}\SpecialCharTok{\textasciitilde{}}\NormalTok{age }\SpecialCharTok{+}\NormalTok{ ejection\_fraction }\SpecialCharTok{+}\NormalTok{ serum\_creatinine }\SpecialCharTok{+}\NormalTok{ time, }\AttributeTok{data=}\NormalTok{train\_data, }\AttributeTok{trControl=}\NormalTok{train\_control, }\AttributeTok{method=}\StringTok{"gbm"}\NormalTok{, }\AttributeTok{verbose=}\DecValTok{0}\NormalTok{)}
\NormalTok{gbm\_pred }\OtherTok{\textless{}{-}} \FunctionTok{predict}\NormalTok{(gbm\_model, }\AttributeTok{newdata =}\NormalTok{ test\_data)}
\end{Highlighting}
\end{Shaded}

Extreme Gradient Boosting:

\begin{Shaded}
\begin{Highlighting}[]
\FunctionTok{set.seed}\NormalTok{(}\DecValTok{467}\NormalTok{)}
\NormalTok{train\_control }\OtherTok{\textless{}{-}} \FunctionTok{trainControl}\NormalTok{(}\AttributeTok{method=}\StringTok{"repeatedcv"}\NormalTok{, }\AttributeTok{number=}\DecValTok{10}\NormalTok{)}
\NormalTok{xgb\_model }\OtherTok{\textless{}{-}} \FunctionTok{train}\NormalTok{(DEATH\_EVENT}\SpecialCharTok{\textasciitilde{}}\NormalTok{., }\AttributeTok{data=}\NormalTok{train\_data, }\AttributeTok{trControl=}\NormalTok{train\_control, }\AttributeTok{method=}\StringTok{"xgbTree"}\NormalTok{)}
\NormalTok{xgb\_pred }\OtherTok{\textless{}{-}} \FunctionTok{predict}\NormalTok{(xgb\_model,}\AttributeTok{newdata =}\NormalTok{ test\_data)}
\end{Highlighting}
\end{Shaded}

Neural Networks:

\begin{Shaded}
\begin{Highlighting}[]
\FunctionTok{set.seed}\NormalTok{(}\DecValTok{888}\NormalTok{)}
\NormalTok{train\_control }\OtherTok{\textless{}{-}} \FunctionTok{trainControl}\NormalTok{(}\AttributeTok{method =} \StringTok{\textquotesingle{}repeatedcv\textquotesingle{}}\NormalTok{, }\AttributeTok{number =} \DecValTok{10}\NormalTok{, }\AttributeTok{classProbs =} \ConstantTok{TRUE}\NormalTok{, }\AttributeTok{verboseIter =} \ConstantTok{FALSE}\NormalTok{, }\AttributeTok{summaryFunction =}\NormalTok{ twoClassSummary, }\AttributeTok{preProcOptions =} \FunctionTok{list}\NormalTok{(}\AttributeTok{thresh =} \FloatTok{0.75}\NormalTok{, }\AttributeTok{ICAcomp =} \DecValTok{3}\NormalTok{, }\AttributeTok{k =} \DecValTok{5}\NormalTok{))}
\NormalTok{nn\_model }\OtherTok{\textless{}{-}} \FunctionTok{train}\NormalTok{(DEATH\_EVENT}\SpecialCharTok{\textasciitilde{}}\NormalTok{anaemia }\SpecialCharTok{+}\NormalTok{ age }\SpecialCharTok{+}\NormalTok{ ejection\_fraction }\SpecialCharTok{+}\NormalTok{ serum\_creatinine }\SpecialCharTok{+}\NormalTok{ time, }\AttributeTok{data =}\NormalTok{ train\_data, }\AttributeTok{method =} \StringTok{\textquotesingle{}nnet\textquotesingle{}}\NormalTok{, }\AttributeTok{preProcess =} \FunctionTok{c}\NormalTok{(}\StringTok{\textquotesingle{}center\textquotesingle{}}\NormalTok{, }\StringTok{\textquotesingle{}scale\textquotesingle{}}\NormalTok{), }\AttributeTok{trControl =}\NormalTok{ train\_control, }\AttributeTok{metric =} \StringTok{"ROC"}\NormalTok{, }\AttributeTok{trace=}\ConstantTok{FALSE}\NormalTok{)}
\NormalTok{nn\_pred }\OtherTok{\textless{}{-}} \FunctionTok{predict}\NormalTok{(nn\_model, }\AttributeTok{newdata =}\NormalTok{ test\_data)}
\end{Highlighting}
\end{Shaded}

\hypertarget{step-4-performance-metrics}{%
\subsection{STEP 4: Performance
Metrics}\label{step-4-performance-metrics}}

To compare the performance of the different models, in order to select a
shortlist of the best models, accuracy, sensitivity, specificity and AUC
are used to compare.

We first write a function to compute the ROC curve and AUC value for
each model to compare the models based on how well they correctly
classify the true positives and true negatives.

\begin{Shaded}
\begin{Highlighting}[]
\NormalTok{roc\_auc\_metrics }\OtherTok{\textless{}{-}} \ControlFlowTok{function}\NormalTok{(model, model\_name)\{}
\NormalTok{  model\_pred }\OtherTok{\textless{}{-}} \FunctionTok{predict}\NormalTok{(model,}\AttributeTok{newdata =}\NormalTok{ test\_data, }\AttributeTok{type=}\StringTok{"prob"}\NormalTok{)}
\NormalTok{  pred }\OtherTok{\textless{}{-}} \FunctionTok{prediction}\NormalTok{(model\_pred[,}\DecValTok{2}\NormalTok{], test\_data}\SpecialCharTok{$}\NormalTok{DEATH\_EVENT)}
\NormalTok{  perf }\OtherTok{\textless{}{-}} \FunctionTok{performance}\NormalTok{(pred, }\AttributeTok{measure =} \StringTok{"tpr"}\NormalTok{, }\AttributeTok{x.measure =} \StringTok{"fpr"}\NormalTok{) }\CommentTok{\#tpr {-} True Positive Rate/Sensitivity}
  \FunctionTok{plot}\NormalTok{(perf,}\AttributeTok{avg=} \StringTok{"threshold"}\NormalTok{, }\AttributeTok{lwd=} \DecValTok{2}\NormalTok{, }\AttributeTok{main=} \FunctionTok{paste}\NormalTok{(}\StringTok{"ROC Curve {-}"}\NormalTok{,model\_name,}\AttributeTok{sep =} \StringTok{" "}\NormalTok{))}
  \CommentTok{\#auc(test\_data$DEATH\_EVENT,model\_pred[,2])}
\NormalTok{  \}}

\NormalTok{models\_list }\OtherTok{\textless{}{-}}\NormalTok{ dplyr}\SpecialCharTok{::}\FunctionTok{lst}\NormalTok{(log\_model,knn\_model,svm\_model,non\_svm\_model,tree\_model,rf\_model,gbm\_model,xgb\_model, nn\_model)}
\FunctionTok{par}\NormalTok{(}\AttributeTok{mfrow=}\FunctionTok{c}\NormalTok{(}\DecValTok{3}\NormalTok{,}\DecValTok{3}\NormalTok{))}
\ControlFlowTok{for}\NormalTok{(model }\ControlFlowTok{in} \FunctionTok{names}\NormalTok{(models\_list))\{}
  \FunctionTok{roc\_auc\_metrics}\NormalTok{(models\_list[[model]],model)}
\NormalTok{\}}
\end{Highlighting}
\end{Shaded}

\includegraphics{Heart-Failure-Project-Markup_files/figure-latex/unnamed-chunk-19-1.pdf}

\begin{Shaded}
\begin{Highlighting}[]
\NormalTok{accuracy\_data }\OtherTok{\textless{}{-}} \ControlFlowTok{function}\NormalTok{(model\_prediction,model\_name)\{}
\NormalTok{  confmat }\OtherTok{\textless{}{-}} \FunctionTok{confusionMatrix}\NormalTok{(model\_prediction, test\_data}\SpecialCharTok{$}\NormalTok{DEATH\_EVENT)}
\NormalTok{  accuracy\_results }\OtherTok{\textless{}{-}} \FunctionTok{round}\NormalTok{(confmat}\SpecialCharTok{$}\NormalTok{overall[[}\StringTok{"Accuracy"}\NormalTok{]]}\SpecialCharTok{*}\DecValTok{100}\NormalTok{, }\AttributeTok{digits=}\DecValTok{2}\NormalTok{)}
  \FunctionTok{return}\NormalTok{(}\FunctionTok{paste}\NormalTok{(model\_name,}\StringTok{" {-} "}\NormalTok{,accuracy\_results,}\StringTok{"\%"}\NormalTok{,}\AttributeTok{sep=}\StringTok{""}\NormalTok{))}
\NormalTok{\}}
\NormalTok{pred\_list }\OtherTok{\textless{}{-}}\NormalTok{ dplyr}\SpecialCharTok{::}\FunctionTok{lst}\NormalTok{(log\_pred,knn\_pred,svm\_pred,non\_svm\_pred,tree\_pred,rf\_pred,gbm\_pred,xgb\_pred,nn\_pred)}
\NormalTok{accuracy\_scores }\OtherTok{\textless{}{-}} \FunctionTok{c}\NormalTok{()}
\ControlFlowTok{for}\NormalTok{(score }\ControlFlowTok{in} \FunctionTok{names}\NormalTok{(pred\_list)) \{}
\NormalTok{    accuracy\_scores }\OtherTok{\textless{}{-}} \FunctionTok{c}\NormalTok{(accuracy\_scores, }\FunctionTok{accuracy\_data}\NormalTok{(pred\_list[[score]], score))}
\NormalTok{\}}
\FunctionTok{print}\NormalTok{(accuracy\_scores)}
\end{Highlighting}
\end{Shaded}

\begin{verbatim}
## [1] "log_pred - 83.05%"     "knn_pred - 86.44%"     "svm_pred - 83.05%"    
## [4] "non_svm_pred - 83.05%" "tree_pred - 86.44%"    "rf_pred - 84.75%"     
## [7] "gbm_pred - 91.53%"     "xgb_pred - 89.83%"     "nn_pred - 86.44%"
\end{verbatim}

\end{document}
